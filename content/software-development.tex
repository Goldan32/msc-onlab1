\chapter{Alkalmazás és hardver fejlesztése}

Ebben a fejezetben bemutatom az eddig elvégzett feladatokat és ezzel együtt a fejlesztés általános folyamatát is egy ilyen heterogén rendszerre. A fejlesztés ilyen rendszerekre közel sem triviális, szerencsére a gyártó általában ad valamilyen kiindulási pontot a kezdéshez, ez a Xilinx esetében is igaz.
\cite{Tutorial}

\section{Vivado base platform}
Ebben a részben a fejlesztés első lépése kerül bemutatásra. A Vivado base platform egyfajta alapot vagy keretet ad az egész projektnek. Ennek a lépésnek a végén rendelkezésre kell állnia egy .xsa fájlnak, ami olyan információkat tartalmaz a hardverről, amiket a fejlesztés további lépéseiben feltétlenül ismerni kell.

\subsection{A base platform részei}
A base platformot a Xilinx Vivado fejlesztő környezet segítségével hoztam létre. Ezen belül a blokkvázlat készítő funkciót kellett használni, amihez sok esetben az IP integrator-ral kellett blokkokat hozzáadni. Az így elkészült blokkvázlatot lehet .xsa fájlként exportálni.

\subsubsection{Zynq MPSoC blokk}
A Vivado base platform talán legfontosabb része a Zynq UltraScale+ MPSoC blokk. Ezt a blokkot készen hozzá tudjuk adni a blokkvázlathoz az IP integrátor segítségével. Hozzáadás után a blokk alapértelmezett beállításokkal kerül hozzáadásra. Az alapértelmezett beállítások függenek a kártya típusától, ezt értelem szerűen a projekt kezdetén be kellett állítani. Az alapértelmezett paramétereket lehet szabadon módosítani, ez saját tervezésű PCB-k esetében lehet érdekes, itt nincs ilyen módosításokra szükség.

Ez a blokk tartalmazza a PS konfigurációit és pin kiosztását. Ez a blokk szolgáltat információt arról, hogy hogyan tudjuk majd PL-ban megtervezett hardvert elérni a PS irányából.
%insert block automation result png

\subsubsection{Órajelek}
Órajelek előállítására rendelkezésre áll a Clocking Wizard nevű IP. Az IP blokk konfigurációja egyszerű, csak meg kell adni, hogy melyik órajelet milyen frekvencián szeretnénk, és az IP elrejti a hardver leírását.

Ebben a feladatban 3 különböző órajelet csinálunk. A rendelkezésre álló órajelek frekvenciái: 100 MHz, 200 MHz és 400 MHz. Ezek közül a későbbiekben bemutatásra kerülő V++ linker a teljes design előállításánál intelligensen tud majd választani. Referencia órajelnek a \mycode{pl\_clk0} nevű órajelet használjuk. Ez a jel a PS felől érkezik, így a blokkvázlaton a Zynq UltraScale+ MPSoC blokk kimeneti jeleként szerepel.

\subsubsection{Reset jelek}
A hardver reset szükségleteinek ellátására is rendelkezésünkre áll egy IP blokk. Mindhárom előállított órajelhez szükség van egy reset jelre is, így három darab Processor System Reset nevű blokkra lesz szükségünk. A három reset blokknak különbözőek az órajelei, de az external reset bemenetük ugyanarra a PS felől érkező reset jelre (\mycode{zynq\_ultra\_ps\_e\_0/pl\_resetn0}) van kötve.

\subsubsection{AXI interface}
A PS és a PL a Xilinx heterogén rendszereiben általában AXI busz interfészen tud kommunikálni. Az IP integrator segítségével hozzá kell adnunk a design-hoz egy AXI Interrupt Controller nevű IP blokkot. 

Ennek a blokknak a konfigurálása során létrehozunk a PS irányából master-nek számító AXI GP (General Purpose) interface-eket. Ezeken keresztül tud majd a szoftver kommunikálni a PL-kal. Ezen kívül a V++ linker automatikusan létrehoz további ilyen irányú AXI interface-eket amiken keresztül a szoftver fel tudja használni a rendelkezésre álló hardveres gyorsító-erőforrásokat.

Szükség van másik irányú kapcsolatra is, tehát a PS-nek slave-ként is kell viselkednie. Ezeken az interface-eken a hardveres gyorsítók a PS-hez tartozó RAM-ot tudják majd elérni.

\subsection{Exportálás}
A base platform már tulajdonképpen készen van, azonban minden hardver design-hoz szükség van egy top modulra. Ezt jelen esetben HDL wrapper-nek hívják, és a Vivado automatikusan le tudja generálni.

A Generate Block Design fázis választása után lehet exportálni az elkészült pre-synthesis design-t az Export Platform opcióval. Az elkészült design-hoz lehetőség van bitstream-et is generálni, erre azonban a feladat során nincs szükség. Az elkészült .xsa fájlt a következő lépesek bemeneteként fogjuk felhasználni, és a PL konfigurációja a fejlesztés során egy későbbi lépésben fog elkészülni.
