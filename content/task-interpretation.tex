\chapter{A feladatkiírás értelmezése, pontosítása}

\section{A feladatkiírás eredeti szövege}

A konvolúciós neurális hálózatok (CNN) alkalmazása igen nagy számítási kapacitást igényel, hoszen minden egyes bemenet kiértékeléséhez rendívül nagy számú MAC műveelt elvégzésére van szükség. Kösznhetően az FPGA-kban alkalmazható nagy mértékű párhuzamosításnak, ezen eszközök kedvező lehetőséget biztosítanak ezen feldolgozási lépés hardveres gyorsítására.

Az önálló labor feladat célja egy látványos, valósidejű CNN demonstráció létrehozása FPGA platform felhasználásával (arc detektálás). A megvalósított rendszer valós idejű kamera képet dolgoz fel, elvégezve a szükséges előfeldolgozási, kiértékelési és utófeldolgozási lépéseket, majd az eredményt megjeleníti kijelzőn.

A fejlesztés a Xilinx Vitis AI szoftverével történik, a megvalósítás platformja pedig vagy az Avnet ULTRA96 kártya vagy a Xilinx ZCU104 fejlesztői kártya. A kiértékelést a Xilinx DPU-v2 DNN IP segítségével végezzük.

\section{A feladatkiírás értelmezése}

Ez a feladat elég szar volt.

Hello this \mycode{code} is neat.

\begin{lstlisting}
#include <stdio.h>

int main()
{
    printf("Hello World\n");
    return 0;
}

\end{lstlisting}