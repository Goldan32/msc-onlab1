\chapter{A feladatkiírás értelmezése, pontosítása}

\section{A feladatkiírás eredeti szövege}

A konvolúciós neurális hálózatok (CNN) alkalmazása igen nagy számítási kapacitást igényel, hiszen minden egyes bemenet kiértékeléséhez rendívül nagy számú MAC művelet elvégzésére van szükség. Köszönhetően az FPGA-kban alkalmazható nagy mértékű párhuzamosításnak, ezen eszközök kedvező lehetőséget biztosítanak ezen feldolgozási lépés hardveres gyorsítására.

Az önálló labor feladat célja egy látványos, valósidejű CNN demonstráció létrehozása FPGA platform felhasználásával (arc detektálás). A megvalósított rendszer valós idejű kamera képet dolgoz fel, elvégezve a szükséges előfeldolgozási, kiértékelési és utófeldolgozási lépéseket, majd az eredményt megjeleníti kijelzőn.

A fejlesztés a Xilinx Vitis AI szoftverével történik, a megvalósítás platformja pedig vagy az Avnet ULTRA96 kártya vagy a Xilinx ZCU104 fejlesztői kártya. A kiértékelést a Xilinx DPU-v2 DNN IP segítségével végezzük.

\section{A feladatkiírás értelmezése}

A feladat során először szükséges megismerkedni a konvolúciós neurális hálózatok (CNN) felépítésével és működésével. A Xilinx Vitis AI szoftveréhez viszonylag sok előre tanított háló áll rendelkezésre. Szükséges ezen nevezetes hálók megismerése, és ezek közül egy kiválasztása, amit a feladat megoldásához felhasználunk majd.

Ezek után meg kell ismerkedni a Xilinx fejlesztői környezetekkel: Vivado, Petalinux és Vitis. Ezek segítségével létre kell hozni egy programot, ami képes neurális hálók kiértékelésére (inference). A feladat megoldására a Xilinx ZCU106 fejlesztői kártyát kell használni.

Először egy olyan programot és konfigurációt kell létrehozni, ahol a hardver csak a neurális hálózat kiértékelését végzi, egy memóriában tárolt képen. A későbbiekben a kép helyett egy folytonos videófolyam képkockáin kell a kiértékelés elvégezni, valamint a hardvernek a folyamat során minél több feladatot át kell vállalnia a szoftvertől.

A végső cél egy olyan program és hardverkonfiguráció, ami a bejövő adatfolyamon neurális hálózat segítségével valós időben tud arcfelismerést végrehajtani.
